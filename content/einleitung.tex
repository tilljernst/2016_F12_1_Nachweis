%%%%%%%%%%%%%%%%%%%%%%%%%%%%%%%%%%%%%%%%%%%%%%%%%%%%%%%%%%%%%%%%%
%_____________ ___    _____  __      __ 
%\____    /   |   \  /  _  \/  \    /  \  Institute of Applied
%  /     /    ~    \/  /_\  \   \/\/   /  Psychology
% /     /\    Y    /    |    \        /   Zürcher Hochschule 
%/_______ \___|_  /\____|__  /\__/\  /    fuer Angewandte Wissen.
%        \/     \/         \/      \/                           
%%%%%%%%%%%%%%%%%%%%%%%%%%%%%%%%%%%%%%%%%%%%%%%%%%%%%%%%%%%%%%%%%
%
% Project     : Seminararbeit
% Title       : 
% File        : einleitung.tex Rev. 00
% Date        : 24.10.2012
% Author      : Till J. Ernst
%
%%%%%%%%%%%%%%%%%%%%%%%%%%%%%%%%%%%%%%%%%%%%%%%%%%%%%%%%%%%%%%%%%

\chapter{Einleitung}\label{chap.einleitung}
TBD: Zitat goethe einrücken
%\glqq Willst Du immer weiterschweifen?
%Sieh, das Gute liegt so nah.
%Lerne nur das Glück ergreifen:
%Denn das Glück ist immer da.\grqq  (J.W. Goethe) \newline
%\glqq Das Vergleichen ist das Ende des Glücks und der Anfang
%der Unzufriedenheit.\grqq (Sören Kierkegaard)


\section{Ausgangslage}\label{ausgangslage}

\begin{itemize}
\item Nennt bestehende Arbeiten/Literatur zum Thema -> Literaturrecherche
\item Stand der Technik: Bisherige Lösungen des Problems und deren Grenzen
\item (Nennt kurz den Industriepartner und/oder weitere Kooperationspartner und dessen/deren Interesse am Thema Fragestellung)
\end{itemize}



\section{Zielsetzung / Aufgabenstellung / Anforderungen}\label{zielsetzung}

\begin{itemize}
\item Formuliert das Ziel der Arbeit
\item Verweist auf die offizielle Aufgabenstellung des/der Dozierenden im Anhang
\item (Pflichtenheft, Spezifikation)
\item (Spezifiziert die Anforderungen an das Resultat der Arbeit)
\item (Übersicht über die Arbeit: stellt die folgenden Teile der Arbeit kurz vor)
\item (Angaben zum Zielpublikum: nennt das für die Arbeit vorausgesetzte Wissen)
\item (Terminologie: Definiert die in der Arbeit verwendeten Begriffe)
\end{itemize}
