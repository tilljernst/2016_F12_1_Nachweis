%%%%%%%%%%%%%%%%%%%%%%%%%%%%%%%%%%%%%%%%%%%%%%%%%%%%%%%%%%%%%%%%%
%_____________ ___    _____  __      __ 
%\____    /   |   \  /  _  \/  \    /  \  Institute of Applied
%  /     /    ~    \/  /_\  \   \/\/   /  Psychology
% /     /\    Y    /    |    \        /   Zürcher Hochschule 
%/_______ \___|_  /\____|__  /\__/\  /    fuer Angewandte Wissen.
%        \/     \/         \/      \/                           
%%%%%%%%%%%%%%%%%%%%%%%%%%%%%%%%%%%%%%%%%%%%%%%%%%%%%%%%%%%%%%%%%
%
% Project     : Latex Vorlage SemArbeit
% Title       : 
% File        : einleitung.tex Rev. 00
% Date        : 24.10.2012
% Author      : Till J. Ernst
%
%%%%%%%%%%%%%%%%%%%%%%%%%%%%%%%%%%%%%%%%%%%%%%%%%%%%%%%%%%%%%%%%%
\chapter{Einleitung}\label{chap.einleitung}
\glsresetall
In den letzten Jahren nahm das Interesse an Kindern und deren soziale Entwicklung gemäss \citeA{Mouratidou:2007} kontinuierlich zu. Der Fokus liegt dabei hauptsächlich bei Kindern auf Stufe Elementarschule \cite{Casey:2001, Gest:2001, Howes:1987, Laosa:1989, Newcomb:1993, Tsiantis:1994, Yeates:1989}. Grösstenteils kann dieses Interesse auf die Tatsache zurückgeführt werden, dass die Interaktion in der Schule mit der Peergruppe eine grosse Rolle bei der sozialen, emotionalen und dem behavioralen Entwicklung spielt\cite{Kupersmidt:1990, Parker:1987, Sandstrom:2003}.\newline
Sozial unsichere Kinder fallen in der Regel nicht als behandlungsbedürftig auf, da sie allem Anschein nach pflegeleicht zu sein scheinen \cite{Petermann:2015}. Ganz im Unterschied zu aggressiven Kindern, bringen sie Erwachsene nicht unter Handlungsdruck. Sie gehen in der Regel nicht aktiv, geschweige denn aggressiv auf andere Kinder oder Erwachsene zu. Vielmehr zeigen sie in der Interaktion mit anderen eine übermässige Schüchternheit, Ängstlichkeit, soziale Unsicherheit sowie Vermeidungsverhalten \cite{Petermann:2015}. Manche Kinder verweigern Sozialkontakte aktiv, andere Kinder wünschen sich Kontakt zu Gleichaltrigen, ohne sich jedoch von ihren Eltern und ihrem häuslichen Umfeld trennen zu können.

\textbf{KJPD Schaffhausen:}
Der Kinder- und Jugendpsychiatrische Dienst Schaffhausen bietet seit einigen Jahren Gruppentherapien für sozial unsichere Kinder an \cite{KJPDW:2016}. Dabei werden Kindern im Alter zwischen sieben und elf Jahren beider Geschlechter die Möglichkeit einer Gruppentherapie angeboten. Die Gruppen bestehen aus rund acht Kindern, in denen auf spielerische Form, mittels Rollen- und Gruppenspielen, an die sozialen Schwierigkeiten herangegangen wird. Den Kindern werden alternative und sicherere Verhaltensmöglichkeiten aufgezeigt und eingeübt. Der rote Faden bildet dabei ein bestimmtes Thema, wie zum Beispiel Freundschaft, Toleranz, Mut, etc. Zu diesem Thema wird dann der Beginn einer Geschichte erzählt, deren Schluss dann von den Kindern in Form eines Gruppentheaters aufgeführt wird. Diese Gruppentherapien finden zweimal jährlich durchgeführt und bestehen aus 12 Terminen an je zwei Stunden. Kinder, die dieses Angebot in Anspruch nehmen, werden in der Regel während einer Abklärung am KJPD vom zuständigen Therapeuten auf das Angebot aufmerksam gemacht, falls dieser Bedarf des jeweiligen Kindes für die Gruppe sieht. Ziel dieser Therapie ist einerseits den Kindern aufzuzeigen, dass sie mit ihren Problemen nicht alleine dastehen. Weiter sollen die Gruppenerlebnisse ein wichtiges Gegengewicht zu den vielen früheren Misserfolgserlebnissen geschaffen werden. Die Gruppe soll als eine Art Übungsfeld dienen, in welchem schwierige Situationen wiederholt und mit der Unterstützung der Leiter ein neuer, besser geeigneter Umgang erprobt und erlebt werden kann \cite{KJPD:2016}. 

\textbf{Soziale Unsicherheit}
Das Verhalten von sozial unsicheren Kindern äussert sich in unterschiedlicher Form. Es kann die Art des Sprechens sein sowie auf die Mimik und Gestik beziehen \cite{Petermann:2015}. Diese Kinder antworten nicht auf Fragen oder nur einsilbig, sprechen leise oder undeutlich. Andere reden zwar viel, ohne dass jedoch ein kommunikativer Austausch stattfindet. Das Reden gleicht einem Selbstgespräch. Viele dieser Kinder sind nicht in der Lage Blickkontakt herzustellen und bei einigen ist auch keine Gefühlsregung im Gesicht erkennbar. Die Klassifikationssysteme DSM und ICD nahmen Angstformen, die für Kinder und Jugendliche spezifisch sind, erst sehr spät auf \cite{Petermann:2015}. 1984 wurden einige Ausdrucksformen der Angst bei Kindern und Jugendlichen im DSM-III zugestanden (Trennungsangst, Kontaktvermeidung und Überängstlichkeit). In der ICD-10 wurde spezifische Kinderängste erst 1991 aufgenommen (Trennungsangst, phobische Störung des Kindesalters und soziale Überempfindlichkeit, später zu sozialer Ängstlichkeit umbenannt). Eine weitere Form des sozial auffälligen Verhaltens, das sich auf neue und unvertraute Situationen bezieht, kann als Schüchternheit bezeichnet werden \cite{Petermann:2015b}. Diese Kinder zeigen ein geringes Selbstbewusstsein, trauen sich wenig zu und reagieren häufig mit Vermeidung. Schüchternheit steht mit geringen sozialen Kompetenzen, vermehrt auftretenden Ängsten und tendenziell mit Depression im Zusammenhang \cite{Karevold:2012}. Der Begriff soziale Unsicherheit umfasst gemäss \citeA{Petermann:2015} Verhaltensweisen, die sich auf Trennungsängste, soziale Angststörung und generalisierte Ängste beziehen. Unter sozialer Unsicherheit versteht \citeA{Petermann:2015} eine verhaltensnahe Sammelbezeichnung, die verschiedenen Angststörungen einschliesst, welche in einem direkten oder indirekten Zusammenhang mit sozialen Anlässen und Situation steht. 

\textbf{Häufigkeit:} Werden verschiedene epidemiologische Studien miteinander verglichen fällt auf, dass je nach Studie unterschiedliche Prävalenzraten bei einzelnen Angststörungen im Kindes- und Jungendalter auftreten. Gemäss \citeA{Petermann:2013} wird eine Auftretenshäufigkeit für alle Angststörungen von 10 Prozent angenommen. Die Studie von \citeA{GrenLandell:2011} können diese Werte für die soziale Angststörung mit einer Prävalenz von 10,6 Prozent bestätigen. Wo hingegen eine norwegische Studie \cite{VanRoy:2009} für die soziale Angst/soziale Phobie eine Prävalenz von 2,3 Prozent angibt. Neuere Übersichten kommen auf die Auftretenshäufigkeit der sozialen Phobie über die Lebensspanne berichten von 10,7 Prozent; 12,3 Prozent bei Frauen und 8,9 Prozent bei Männern \cite{Kessler:2012}. Dabei ist zu beachten, dass sich die soziale Phobie schon zwischen dem 13. und 15. Lebensjahr als stabile Störung herausbildet und stabil bleibt \cite{Petermann:2015}. 

\textbf{Stand der Forschung:} In einer Kontrollgruppenstudie von 
\citeA{Ortbandt:2009} wurde die Wirksamkeit von Gruppentherapie mit sozial unsicheren Kindern empirisch überprüft. Diese Studie hatte zum Ziel, das ängstliche Verhalten der Kinder durch die Teilnahme an der Therapie zu untersuchen und ob dieses Verhalten sich verringert sowie über einen Zeitraum von sechs Monaten stabil bleibt. An der Studie nahmen 19 Kinder (10 Mädchen, 9 Jungen) im Alter von 7;2 bis 12;7 Jahren teil (Durchschnitt 9;7 Jahre). Die Kinder erfüllten die Kriterien für eine der nachfolgenden Angststörungen nach ICD-10: Trennungsangst (F93.0), soziale Ängstlichkeit (F93.2), soziale Phobie (F40.1) oder generalisierte Angststörung (F93.80). Für die Rekrutierung wurden verschiedene Einrichtungen eibezogen (z.B. Krankenhäuser, Kinderzentren, Fachärzte sowie Kinder- und Jugendpsychiatrie, Schulen und Erzeihungsberatungsstellen). Die dadurch gewonnenen Kinder wurden zufällig in eine Interventionsgruppe (10 Kinder) und eine Wartegruppe (9 Kinder) zugeteilt. Die Studie untersuchte vier Zeitpunkte, an denen Angaben der Eltern und Lehrkräfte sowie Auskünfte der Kinder erhoben wurde. Zusätzlich wurde bei jedem Zeitpunkt das Vorliegen von Angststörungen und einer depressiven Störung mit dem Diagnostik-System für psychische Störungen im Kindes- und Jugendalter nach ICD-10 und DSM-IV geprüft. Die Studie konnte im Vergleich zwischen der Therapiegruppe und der Kontrollgruppe eine kurzfristige Wirksamkeit ermitteln. Aus Sicht der Eltern wurden ängstliche, aber auch depressive Verhaltensweisen durch die Behandlung der Angststörung deutlich verringert. Im Einzelnen wurden in der Interventionsgruppe deutliche Erfolge im Bereich Trennungsangst gemessen. Jedoch nur einen mittleren Effekt im Bereich der sozialen Angst. Die Autoren gehen davon aus, da die Werte im Vergleich zur Trennungsangst sehr hoch ausgeprägt waren, dass das Vorgehen bei stark ausgeprägten Angststörungen intensiviert werden muss. Ebenso konnte die Studie durch die Behandlung der Angststörung einen positiven Effekt beim depressiven Verhalten nachweisen. Durch den Vergleich der verschiedenen Messpunkte (Prä, Post und Follow up) konnten Aussagen über die kurz- und langfristige Wirksamkeit der Therapie gewonnen werden \cite{Moller:2011}. Sechs Monate nach der Therapie konnte festgestellt werden, dass sich das ängstliche Verhalten der Kinder aus Sicht der Eltern, Lehrkräfte und Kinder verglichen mit der Situation vor der Therapie stark reduziert hatten. Dieser Effekt zeigte sich hauptsächlich in einem verbesserten Sozialkontakt, einer verstärkten Kooperation, selbständigerem Lernen und einer angemessenen Selbstbehauptung. Die Kinder gaben sogar an, dass sich der positive Effekt bei sozialen Situationen weniger stark zu vermeiden über die sechs Monate gegenüber der Postmessung weitere verstärkte. Dies spricht dafür, dass die Kinder die neu erlernten Verhaltensweisen erst allmählich in verschiedenen Lebensbereichen erfolgreich einsetzten und dies immer besser gelang \cite{Moller:2011}. Für die Praxis schlagen die Autoren vor \cite{Petermann:2015}, dass die Intervention, das Training mit sozial unsicheren Kindern, auf eine grössere Anzahl von Sitzungen verteilt werden sollte. Wird auf der Basis der Stichproben die Übereinstimmung von Eltern- und Lehrerurteil bezüglich der sozialen Angst verglichen, egeben sich geringe Korrelationen von maximal 0.27 \cite{Moller:2011}. Für die klinische Praxis wird deshalb empfohlen, verschiedenen Informationsquellen zu nutzen (Selbst-, Eltern- und Expertenurteil). Zudem liefert die Gruppenstudie von \citeA{Ortbandt:2009} Hinweise darauf, dass die Therapie bei stärker ausgeprägten Angststörungen intensiviert werden muss.







