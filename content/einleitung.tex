%%%%%%%%%%%%%%%%%%%%%%%%%%%%%%%%%%%%%%%%%%%%%%%%%%%%%%%%%%%%%%%%%
%_____________ ___    _____  __      __ 
%\____    /   |   \  /  _  \/  \    /  \  Institute of Applied
%  /     /    ~    \/  /_\  \   \/\/   /  Psychology
% /     /\    Y    /    |    \        /   Zürcher Hochschule 
%/_______ \___|_  /\____|__  /\__/\  /    fuer Angewandte Wissen.
%        \/     \/         \/      \/                           
%%%%%%%%%%%%%%%%%%%%%%%%%%%%%%%%%%%%%%%%%%%%%%%%%%%%%%%%%%%%%%%%%
%
% Project     : Latex Vorlage SemArbeit
% Title       : 
% File        : einleitung.tex Rev. 00
% Date        : 24.10.2012
% Author      : Till J. Ernst
%
%%%%%%%%%%%%%%%%%%%%%%%%%%%%%%%%%%%%%%%%%%%%%%%%%%%%%%%%%%%%%%%%%
\chapter{Einleitung}\label{chap.einleitung}
\glsresetall
% Kapitel Ausgangslage
\section{Hintergrund}\label{hintergrund.section}
In den letzten Jahren nahm das Interesse an Kindern und deren soziale Entwicklung gemäss \cite{Mouratidou:2007} kontinuierlich zu. Der Fokus liegt dabei hauptsächlich bei Kindern auf Stufe Elementarschule \cite{Casey:2001, Gest:2001, Howes:1987, Laosa:1989, Newcomb:1993, Tsiantis:1994, Yeates:1989}. Grösstenteils kann dieses Interesse auf die Tatsache zurückgeführt werden, dass die Interaktion in der Schule mit der Peergruppe eine grosse Rolle bei der sozialen, emotionalen und dem behavioralen Enwticklung spielt\cite{Kupersmidt:1990, Parker:1987, Sandstrom:2003}.\newline
Sozial unsichere Kinder fallen in der Regel nicht als behandlungsbedürftig auf, da sie allem Anschein nach pflegeleicht zu sein scheinen \cite{Petermann:2015}. Ganz im Unterschied zu aggressiven Kindern, bringen sie Erwachsene nicht unter Handlungsdruck. Sie gehen in der Regel nicht aktive, geschweige denn aggressiv auf andere Kinder oder Erwachsene zu. Vielmehr zeigen sie in der Interaktion mit anderen eine übermässige Schöchternheit, Ängstlichkeit, soziale Unsicherheit sowie Vermeidungsverhalten \cite{Petermann:2015}. Manche Kinder verweigern Sozialkontakte aktiv, andere Kinder wünschen sich Kontakt zu Gleichaltrigen, ohne sich jedoch von ihren Eltern und ihrem häuslichen Umfeld trennen zu können.\newline
Der Kinder- und Jugendpsychiatrische Dienst Schaffhausen (KJPD) bietet seit einigen Jahren Gruppentherapien für sozial unsichere Kinder an \cite{KJPD:2016}. Dabei werden Kindern im Alter zwischen sieben und elf Jahren beider Geschlechter die Möglichkeit einer Gruppentherapie angeboten. Die Gruppen bestehen aus rund acht Kindern, in denen auf spielerische Form, mittels Rollen- und Gruppenspielen, an die sozialen Schwierigkeiten herangegangen wird. Den Kindern werden alternative und sicherere Verhaltensmöglichkeiten aufgezeigt und eingeübt. Der rote Faden bildet dabei ein bestimmtes Thema, wie zum Beispiel Freundschaft, Toleranz, Mut, etc. Zu diesem Thema wird dann der Beginn einer Geschichte erzählt, deren Schluss dann von den Kindern in Form eines Gruppentheaters aufgeführt wird. Diese Gruppentherapien finden zweimal jährlich durchgeführt und bestehen aus 12 Terminen an je zwei Stunden. Kinder, die dieses Angebot in Anspruch nehmen, werden in der Regel während einer Abklärung am KJPD vom zuständigen Therapeuten auf das Angebot aufmerksam gemacht, falls dieser Bedarf des jeweiligen Kindes für die Gruppe sieht. Ziel dieser Therapie ist einerseits den Kindern aufzuzeigen, dass sie mit ihren Problemen nicht alleine dastehen. Weiter sollen die Gruppenerlebnisse ein wichtiges Gegengewicht zu den vielen früheren Misserfolgserlebnissen geschaffen werden. Die Gruppe soll als eine Art Übungsfeld dienen, in welchem schwierige Situationen wiederholt und mit der Unterstützung der Leiter ein neuer, besser geeigneter Umgang erprobt und erlebt werden kann \cite{KJPD:2016}. 
\section{Soziale Unsicherheit}\label{sozialeUnsicherheit.section}
\section{Stand der Forschung}\label{standDerForschung.section}
Hallo Text \cite{Petermann:2015}.
Hallo Text \cite{Gehrig:2014}.
Hallo Text \cite{Mouratidou:2007}.
Hallo Text \cite{Petermann:1989}.



