%%%%%%%%%%%%%%%%%%%%%%%%%%%%%%%%%%%%%%%%%%%%%%%%%%%%%%%%%%%%%%%%%
%_____________ ___    _____  __      __ 
%\____    /   |   \  /  _  \/  \    /  \  Institute of Applied
%  /     /    ~    \/  /_\  \   \/\/   /  Psychology
% /     /\    Y    /    |    \        /   Zürcher Hochschule 
%/_______ \___|_  /\____|__  /\__/\  /    fuer Angewandte Wissen.
%        \/     \/         \/      \/                           
%%%%%%%%%%%%%%%%%%%%%%%%%%%%%%%%%%%%%%%%%%%%%%%%%%%%%%%%%%%%%%%%%
%
% Project     : Latex Vorlage SemArbeit
% Title       : 
% File        : einleitung.tex Rev. 00
% Date        : 24.10.2012
% Author      : Till J. Ernst
%
%%%%%%%%%%%%%%%%%%%%%%%%%%%%%%%%%%%%%%%%%%%%%%%%%%%%%%%%%%%%%%%%%
\chapter{Fragestellung}\label{chap.fragestellung}
\glsresetall
\textbf{Ziel.} Ziel dieser Studie ist es, die Wirksamkeit der Gruppentherapie für sozial unsichere Kinder am KJPD Schaffhausen zu überprüfen. Dabei soll das eigens am KJPD Schaffhausen entwickelte und seit einigen Jahren eingesetzte Programm verwendet werden. Dabei wird wie in der Studie von \citeA{Ortbandt:2009} überprüft und davon ausgegangen, dass das ängstliche Verhalten der Kinder durch die Teilnahme an der Gruppentherapie reduziert wird.

\begin{description}
  \item[Fragestellungen]~\par
  \begin{enumerate}
      \item Können die Einschränkungen, die mit einer Angststörung verbunden sind, in der Schule und Freizeit (z.B. beeinträchtigte Leistungsfähigkeit, soziale Isolation) durch die Teilnahmen an der Gruppentherapie vermindert werden?
      \item Verringern sich die depressiven Symptome, die im Rahmen einer Angststörung verbunden sein können \cite{Essau:2002} durch die Wirksamkeit der Gruppentherapie?
      \item Sind die durch die Intervention gewonnenen Verbesserungen bezüglich Verhalten, Einschränkungen und Wirksamkeit nachhaltig (nach 6 Monaten nach der Teilnahme an der Gruppentherapie nachweisbar)?
  \end{enumerate}
\end{description}  

\begin{description}
  \item[Hypothesen]~\par
  \begin{enumerate}
      \item Durch die Teilnahme an der Gruppentherapie werden Einschränkungen, die mit einer Angststörung verbunden, sind signifikant verringert.
      \item Depressive Symptome, die im Rahmen einer Angststörung auftreten können, sollten sich durch die Wirksamkeit der Gruppentherapie signifikant verringern.
      \item Es wird davon ausgegangen, dass die durch die Gruppentherapie erreichten Verbesserungen auch sechs Monate nach Abschluss der Intervention noch nachweisbar sind.
  \end{enumerate}
\end{description}  