%%%%%%%%%%%%%%%%%%%%%%%%%%%%%%%%%%%%%%%%%%%%%%%%%%%%%%%%%%%%%%%%%
%_____________ ___    _____  __      __ 
%\____    /   |   \  /  _  \/  \    /  \  Institute of Applied
%  /     /    ~    \/  /_\  \   \/\/   /  Psychology
% /     /\    Y    /    |    \        /   Zürcher Hochschule 
%/_______ \___|_  /\____|__  /\__/\  /    fuer Angewandte Wissen.
%        \/     \/         \/      \/                           
%%%%%%%%%%%%%%%%%%%%%%%%%%%%%%%%%%%%%%%%%%%%%%%%%%%%%%%%%%%%%%%%%
%
% Project     : Latex Vorlage SemArbeit
% Title       : 
% File        : einleitung.tex Rev. 00
% Date        : 24.10.2012
% Author      : Till J. Ernst
%
%%%%%%%%%%%%%%%%%%%%%%%%%%%%%%%%%%%%%%%%%%%%%%%%%%%%%%%%%%%%%%%%%
\chapter{Fragestellung}\label{chap.fragestellung}
\glsresetall
Abgekupfert von \cite{Ortbandt:2009}: \newline
In der vorliegenden Studie wird die Wirksamkeit des Trainings mit sozial unsicheren Kindern (Petermann \& Petermann, 2006) überprüft und davon ausgegangen, dass das ängstliche Verhalten der Kinder durch die Teilnahme am Training reduziert wird. Die mit einer Angststörung verbundenen Einschränkungen in Schule und Freizeit (z. B. beeinträchtigte Leistungsfähigkeit, soziale Isolation) können durch eine erfolgreiche Teilnahme am Training vermindert werden. Da die Beeinträchtigungen im Rahmen einer Angststörung (z.B. unter sozialer Isolation leiden) mit depressiven Symptomen verbunden sein können (Beesdo et al., 2007; Chavira, Stein, Bailey \& Stein, 2004; Essau, Conradt \& Petermann, 2000, 2002; Ford, Goodman \& Meltzer, 2003), sollten sich auch diese durch die Wirksamkeit des Trainings verringern.