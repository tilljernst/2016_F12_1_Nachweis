%%%%%%%%%%%%%%%%%%%%%%%%%%%%%%%%%%%%%%%%%%%%%%%%%%%%%%%%%%%%%%%%%
%_____________ ___    _____  __      __ 
%\____    /   |   \  /  _  \/  \    /  \  Institute of Applied
%  /     /    ~    \/  /_\  \   \/\/   /  Psychology
% /     /\    Y    /    |    \        /   Zuercher Hochschule 
%/_______ \___|_  /\____|__  /\__/\  /    fuer Angewandte Wissen.
%        \/     \/         \/      \/                           
%%%%%%%%%%%%%%%%%%%%%%%%%%%%%%%%%%%%%%%%%%%%%%%%%%%%%%%%%%%%%%%%%
%
% Project     : Seminararbeit
% Title       : 
% File        : sozialeMedien.tex Rev. 00
% Date        : 10.10.2012
% Author      : Till J. Ernst
%
%%%%%%%%%%%%%%%%%%%%%%%%%%%%%%%%%%%%%%%%%%%%%%%%%%%%%%%%%%%%%%%%%

\chapter{Soziale Medien}\label{chap.sm}
\glsresetall
In einem ersten Teil wird auf die Begrifflichkeit von \gls{sm} eingegangen. Danach folgt eine Klassifikation, geschlossen wird die Arbeit mit einem kurzen Blick in die Zukunft.\newline
Da es sich beim Thema \gls{sm} um eine Technologie handelt, die vorwiegend in englischer Sprache verfasst und definiert ist, kann durch die Übersetzung die gendergerechte Form nicht immer berücksichtigt werden. Wo immer ein Rückschluss auf eine männliche Form vorgenommen werden kann, ist implizit auch die weibliche Form gemeint und umgekehrt.  Nicht immer ist zu einem englischen Fachbegriff ein korrelierendes deutsches Wort vorhanden. In diesen Fällen wird das englische Wort ohne Übersetzung übernommen.
%UK Zentrale Begriffe
\section{Begriffserklärung}\label{sec.begriff}
Gemäss \citeA{Sjurts:2011} ist \gls{sm} (engl. social media) ein Sammelbegriff für internet-basierte mediale Angebote, die auf sozialen Interaktionen basieren. Die Angebote, auch Applikationen genannt, bauen auf den Ideologien und den technischen Möglichkeiten von Web 2.0 auf \cite{Kaplan:2010} und ermöglichen das Erstellen und Austauschen von nutzererzeugten Inhalten (engl. user generated content).\newline
Web 2.0 und nutzerzeugte Inhalte bilden dabei die Grundkonzepte von \gls{sm}. Der Begriff und die Bedeutung Web 2.0 wurde 2004 eingeführt. Ziel war es, die Nutzung des weltweiten Netzes (engl. World Wide Web, kurz Web oder WWW) zwischen Personen, die Software entwickeln und Personen, die sie nutzen, neu zu definieren (ebda.,2010). Web 2.0 stellt dabei eine Plattform dar, auf der nicht mehr einzelne Personen für den Inhalt und die Verteilung von Angeboten verantwortlich sind, sondern eine Vielzahl von beteiligten Personen gemeinsam. Neben dieser Plattform, die als technologisches Fundament gilt, kann der nutzerzeugte Inhalt als die Summe aller Möglichkeiten betrachtet werden, wie die beteiligten Personen die \gls{sm} nutzen können. Dazu müssen gemäss \citeA{OECD:2007} drei grundlegende Anforderungen erfüllt sein: Erstens müssen nutzererzeugte Inhalte auf einer öffentlich publizierten Webseite zugänglich sein oder mittels Sozialer Netzwerke für eine ausgewählte Gruppe zur Verfügung stehen, was z.B. E-Mail ausschliesst. Zweitens muss der Inhalt einem gewissen Anteil an kreativem Aufwand genügen, was eine reine Kopie eines bestehenden Inhalts bereits erfüllt und drittens muss der Inhalt ausserhalb eines professionellen Umfeldes, auf dem Hintergrund eines kommerziellen Marktes, erzeugt worden sein. Nutzerzeugte Inhalte wurden bereits vor dem Web 2.0, in den frühen 1980er Jahren beschrieben. Aber erst mit den Möglichkeiten von Web 2.0, technologischer Treiber (z.B. Breitband-Internetanschluss), ökonomischer Veränderung (z.B. frei zugängliche Angebote für die Generierung von Inhalten) und soziologischer Faktoren (z.B. Entstehung einer Generation mit erheblichem technischen Fachwissen, vgl. \textquotedblleft digital natives\textquotedblright) konnten die gesamten Möglichkeiten angewendet werden und umschreiben den heutigen Begriff \gls{sm}.\newline 
Eine weitere ergänzende Definition gemäss \citeA{Ahlqvist:2008} unterteilt \gls{sm} in drei Hauptkategorien: Inhalt, Gemeinschaft und Web 2.0. Inhalt bezeichnet den von Personen generierten Inhalt, der unterschiedlicher Art sein kann (z.B. Bilder, Fotos, Videos, Kommentare, Standortinformationen, etc.). Gemeinschaft ist bereits im englischen Wort \textquotedblleft social\textquotedblright \ enthalten und bezeichnet das Bereitstellen von Inhalt auf dem Web und das Interesse am Inhalt Anderer. \gls{sm} wird dazu verwendet direkt oder mittels aufgezeichneten Daten miteinander zu kommunizieren. Die Webtechnologien und die Anwendungen, die es Personen ermöglicht über das Web zu kommunizieren, bezeichnet die dritte Kategorie Web 2.0. \newline
\gls{sm} vereint somit mobile und web-basierte Technik, um nutzererzeugten Inhalt auf einer flexiblen Plattform zwischen Einzelpersonen und Gruppen zu teilen, zu erzeugen, untereinander zu bewerten und zu verändern \cite{Kietzmann:2011}.
 
%UK Klassifizierung
\section{Klassifizierung}\label{sec.klassifiezierung}
\gls{sm} beinhalten eine riesige Menge an Anwendungen und Möglichkeiten. Eine Klassifizierung kann dabei helfen, den Überblick zu bewahren. \citeA{Kietzmann:2011} teilt die \gls{sm} in separate Blöcke ein, die sich bienenwabenartig aneinanderreihen lassen. Die funktionalen Blöcke setzen sich zusammen aus Identität, Konversation, gemeinsame Nutzung, Anwesenheit, Beziehung, Ansehen und Gruppe. Jeder Block untersucht dabei eine spezifische Facette einer Person, die \gls{sm} verwendet und erlaubt die Funktionen von \gls{sm} differenziert in verschiedene Ebenen zu unterteilen. Die Blöcke ziehen keine scharfen Grenzen untereinander und schliessen sich gegenseitig nicht aus. Vielmehr sind die Übergänge fliessend. Für eine Einteilung müssen auch nicht alle Blöcke verwendet werden. Diese Einteilung anhand eines solchen Rahmensystems nach \citeauthor{Kietzmann:2011} ermöglicht \gls{sm} besser zu verstehen und welchen Nutzen sie haben (persönlich und wirtschaftlich). \newline
Ein weiteres Klassifizierungsschema gemäss \citeA{Kaplan:2010} basiert auf medienwissenschaftlichen und soziologischen Theorien. Auf dieses Schema wird im Folgenden spezifischer eingegangen, da in dieser Einteilung Konzepte verwendet werden, die im Kapitel \ref{chap.einfluss} - \nameref{chap.einfluss} wiederkehren. Aus medienwissenschaftlicher Sicht spielt die Theorie der sozialen Präsenz (engl. social presence) von \citeA{Short:1976} und die Medienreichhaltigkeitstheorie (engl. media richness) von \citeA{Daft:1986} eine Rolle. Die Theorie der sozialen Präsenz geht davon aus, dass je höher die soziale Präsenz eines Kommunikationsmediums ist, desto höher ist der gegenseitige Einfluss auf die Kommunikationspartner und Kommunikationspartnerinnen. Medienreichhaltigkeitstheorien nach \citeA{Daft:1986} basieren auf der Annahme, dass das Hauptziel jeglicher Kommunikation die Auflösung von Mehrdeutigkeit und die Reduktion von Unsicherheit sein sollte. Im Hinblick auf die soziologischen Theorien, die eine wichtige Rolle im Bereich der \gls{sm} spielen, ist das Konzept der Selbstdarstellung (engl. self-presentation) nach \citeA{Goffman:1959} und die Theorie der Selbstoffenbarung (engl. self-disclosure) nach \citeA{Schau:2003}. Das Konzept der Selbstdarstellung beschreibt, dass in jeglicher Form der sozialen Interaktion die beteiligten Personen versuchen einen möglichst kontrollierten Eindruck von sich selber abzugeben. Üblicherweise geht dieser Vorgang mit einer gewissen Selbstoffenbarung einher, welches die bewusste oder unbewusste Offenbarung von persönlichen Informationen beinhaltet. Beide Dimensionen kombiniert, die der medienwissenschaftlichen und die der soziologischen,  führen zu einer Klassifikation von \gls{sm}, wie in folgender Tabelle gemäss \citeA{Kaplan:2010} dargestellt:	
%Tabelle >{\columncolor{gray}}
\begin{table}[ht] \centering
	\caption{Klassifizierung}
	\begin{tabular}[t]{|m{27mm} r|m{28mm}|m{28mm}|m{35mm}|} \hline 
		\multicolumn{2}{|c|}{} & \multicolumn{3}{c|}{\textbf{Soziale Präsenz / Medienreichhaltigkeit}} \\
		\multicolumn{2}{|c|}{} & \cellcolor{gray} \textbf{Low} & \cellcolor{gray} \textbf{Medium} & \cellcolor{gray} \textbf{High} \\ 
			\hline 
		\multirow{2}{27mm}{\textbf{Selbst- darstellung und Selbstoffenbarung}} & \cellcolor{gray}\textbf{High} & Blogs & Soziale Netzwerke (z.B. Facebook) & Virtuelle Welten (z.B. Second Life) \\ 
			\cline{2-5}
		& \cellcolor{gray}\textbf{Low} & Gemeinschafts- projekte (z.B. Wikipedia) & Content-Communities (z.B. YouTube) & Virtuelles Rollenspiel (z.B. World of Warcraft) \\ 
			\hline
	\end{tabular}
	\label{tab:Klassifiaktion}
\end{table}
 
Im Hinblick auf die Soziale Präsenz und die Medienreichhaltigkeit wirken sich die Gemeinschaftsprojekte und Blogs am wenigsten auf diese Konzepte aus, da diese oft nur einen textbasierten Aufbau haben und einen relativ simplen Datenaustausch zulassen. Auf der nächsten Ebene folgen die Content-Communities und Soziale Netze, welche zusätzlich zum Austausch von eher textbasierten Daten auch Videos und andere Medien anbieten. Auf der höchsten Ebene siedeln sich die virtuellen Rollenspiele und die virtuellen Welten an, welche alle Kommunikationsfaktoren, die bei einer Kommunikation von Angesicht zu Angesicht vorhanden sind, in einer virtuellen Welt abzubilden versuchen. \newline 
Auf die Selbstdarstellung und die Selbstoffenbarung bezogen, werden Blogs höher eingestuft als die Gemeinschaftsprojekte, da diese auf einen spezifischen Inhalt gerichtet sind. Ähnlich verhält es sich mit den Sozialen Netzen und den Content-Communities. Abschliessend benötigen virtuelle Welten einen höheren Selbstoffenbarungsgrad, als dies die virtuellen Rollenspiele verlangen, da letztere starke Vorschriften betreffend dem Verhalten ihrer Spieler und Spielerinnen vorschreiben.

%SubSec Social Network
\section{Soziale Netzwerke und Co}\label{sec.sn}
In diesem Kapitel werden die wichtigsten Begriffe, die im Zusammenhang mit \gls{sm} und dieser Arbeit auftreten, kurz beschrieben:\par  
\textbf{Blogs:} Blogs sind gemäss \citeA{Kaplan:2010} die ursprünglichste Form von \gls{sm}. Sie sind eine spezielle Form von Webseiten, die üblicherweise chronologisch geordnete Einträge beinhalten. Aus Sicht von \gls{sm} sind sie das Pendant zu den persönlichen Webseiten von Privatpersonen. Sie werden in verschiedensten grafischen Formen angeboten und beinhalten unterschiedliche Themen, von persönlichen Tagebüchern, die das Leben eines Autors wiedergeben, bis hin zu Einträgen und Information zu einem spezifischen Themengebiet. Blogs werden in der Regel von einer Person mit Inhalten versorgt, besitzen aber die Möglichkeit, Kommentare zu den Inhalten zu hinterlassen. \par 
\textbf{Gemeinschaftsprojekte} (engl. content communities)\textbf{:} Der Hauptnutzen von Gemeinschaftsprojekten ist das Verteilen von medialen Inhalten zwischen Personen (z.B.: Text, Fotos, Videos, PowerPoint-Präsentation, etc.)(ebda.,2010). Personen sind nicht verpflichtet ein persönliches Konto beim Anbieter zu eröffnen, um solche Dienste zu nutzen und Inhalte zu beziehen. Aus unternehmenstechnischer Sicht tragen Gemeinschaftsprojekte das Risiko, für die Verteilung von geschützten Inhalten missbraucht zu werden (z.B.: urheberrechtlich geschützte Kinofilme). \par 
\textbf{Soziale Netzwerke} (engl. social networking sites)\textbf{:} Unter Sozialen Netzwerken versteht man auf dem Internet erreichbare Seiten, auf denen Personen ein Konto mit einem eigenen Profil erstellen können. Mit einem Benutzerzugang können Freunde und Bekannte eingeladen werden, das eigene Profil zu betrachten, gegenseitig Nachrichten zu senden und Daten auszutauschen. Diese persönlichen Profile können verschiedenste Arten von Daten beinhalten wie Fotos, Videos, Musikdateien und Blogs. Gemäss Wikipedia, welches zu den Gemeinschaftsprojekten gehört, gehört das 2004 von Mark Zuckerberg an der Harvard Universität gegründete Facebook mit über einer Milliarde Nutzern \cite{Facebook:2012} zu den weltweit grössten Sozialen Netzwerken.\par 
\textbf{Virtuelle Rollenspiele} (engl. virtual game worlds)\textbf{:} Virtuelle Rollenspiele finden in einer fiktiven Welt statt, in der die spielenden Personen mittels personalisiertem Avatar (virtuelle Figur) untereinander interagieren können. Diese Welt stellt wohl die am weitesten umgesetzte Erscheinungsform von \gls{sm} dar. Das zur Zeit grösste Rollenspiel 'World of Warcraft' zählt über 10 Millionen Benutzerabonnements weltweit \cite{Blizzard:2012}.\par 
\textbf{Virtuelle Welten} (engl. virtual social worlds)\textbf{:} Virtuelle Welten erlauben den Bewohnerinnen und Bewohnern, sich wie in der realen Welt zu bewegen. Analog zu den oben erwähnten Rollenspielen können sich die Personen in Form von einem persönlichen Avatar in dieser Welt bewegen und mit anderen Avataren in einer virtuellen dreidimensionalen Welt interagieren. Jede erdenkliche Form der Selbstdarstellung kann in dieser Welt gelebt werden \cite{Kaplan:2010}. Gemäss \citeA{Gridsurvey:2012} sind zur Zeit über 28 Millionen registrierte Benutzer in der wohl bekanntesten virtuellen Welt 'Second Life' angemeldet.
%UK Zukunft
\section{Ausblick}\label{sec.ausblick}
Die heutige Soziale Medien Revolution, wie sie von \citeA{Kaplan:2012} genannt wird, kann als Evolution zurück zu den ursprünglichen Wurzeln des Internets betrachtet werden. Eine Plattform, die es den Benutzerinnen und Benutzern ermöglich, untereinander Informationen auf einem einfachen Weg auszutauschen. Auch für Firmen ist der Umgang mit \gls{sm} unumgänglich geworden \cite{Kietzmann:2011}, da Kunden den aktiven Umgang mit den Firmen über \gls{sm} suchen und sich selber einbringen wollen.\newline
Das starke Aufkommen von mobilen Endgeräten \cite{Kaplan:2012}, beeinflusst das Verhalten und den Umgang mit \gls{sm}. Mit einem mobilen Endgerät, wie zum Beispiel mittels Smartphone (iPhone, etc.), wird nicht mehr nur der Status übermittelt, sondern auch der exakte Standort der Person. Dies führt zu weiteren Möglichkeiten im Umgang mit (mobilen) Sozialen Medien. \newline
  





