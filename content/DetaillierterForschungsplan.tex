%%%%%%%%%%%%%%%%%%%%%%%%%%%%%%%%%%%%%%%%%%%%%%%%%%%%%%%%%%%%%%%%%
%_____________ ___    _____  __      __ 
%\____    /   |   \  /  _  \/  \    /  \  Institute of Applied
%  /     /    ~    \/  /_\  \   \/\/   /  Psychology
% /     /\    Y    /    |    \        /   Zürcher Hochschule 
%/_______ \___|_  /\____|__  /\__/\  /    fuer Angewandte Wissen.
%        \/     \/         \/      \/                           
%%%%%%%%%%%%%%%%%%%%%%%%%%%%%%%%%%%%%%%%%%%%%%%%%%%%%%%%%%%%%%%%%
%
% Project     : Latex Vorlage SemArbeit
% Title       : 
% File        : einleitung.tex Rev. 00
% Date        : 24.10.2012
% Author      : Till J. Ernst
%
%%%%%%%%%%%%%%%%%%%%%%%%%%%%%%%%%%%%%%%%%%%%%%%%%%%%%%%%%%%%%%%%%
\chapter{Detaillierter Forschungsplan}\label{chap.forschungsplan}
\glsresetall
\textbf{Design.} 
Für die Studie wird ein experimentelles Design mit einer Versuchsgruppen (Double-Pretest-Follow-up-Design) über den Zeitraum von mindestens zwei Jahren gewählt. Im KJPD werden zwei Gruppentherapien pro Jahr mit etwa 7 bis 8 Teilnehmer angeboten. Ziel ist es, die Wirksamkeit der Gruppentherapie (Intervention) und deren Langzeitfolge zu untersuchen. Die Intervention besteht aus einem am KJPD Schaffhausen angebotenen Gruppentherapi-Angebot, das über den Zeitraum von vier Monaten an 12 Sitzungen stattfindet. Dabei findet für jede Gruppe vor und nach der Intervention (X) zwei Erhebungszeitpunkte (O\textsubscript{1-4}) statt. Die Messzeitpunkte (O\textsubscript{1 und 2}) vor der Intervention dienen der Absicherung für die Wirksamkeit bei den Wartegruppen. Der Messzeitpunkt (0\textsubscript{3}) unmittelbar nach der Intervention dient zur Wirksamkeitsüberprüfung der Gruppentherapie. Der vierte und letzte Messzeitpunkt(0\textsuperscript{3}) erfolgt ein halbes Jahr nach der Intervention und dient der Überprüfung, ob die gefundenen Effekte nachhaltig und sind.
\begin{center}
    \framebox{
        O\textsubscript{1}    \hspace{3mm} O\textsubscript{2} \hspace{3mm} X \hspace{3mm} O\textsubscript{3} \hspace{3mm} O\textsubscript{4}
    }
    
    O = Messung (=Observation); X Intervention
\end{center}
Für die Diagnose und die daraus folgende Aufnahme in die Gruppe wird ein zweistufiges Vorgehen gewählt. Als erstes sollen die Eltern mit Hilfe der Diagnosecheckliste aus dem Training mit sozial unsicheren Kindern \cite{Petermann:2015} das Verhalten ihres Kindes gemäss den ICD-10 Kriterien beurteilen (Trennungsangst, soziale Ängstlichkeit, generalisierte Angststörung und depressive Stimmung). Anschliessend werden diejenigen Kinder, die den Diagnosekriterien einer Angststörung entsprechen, von den behandelnden Psychologinnen und Psychologen am KJPD exploriert. Wird die Diagnose bestätig, werden die Kinder in die Gruppe aufgenommen. 

Für die Überprüfung der Wirksamkeit des Trainings wird neben Angaben der Eltern (Fremdeinschätzung) auch Auskünfte der Kinder (Selbsteinschätzung) herangezogen. Das Vorliegen von Angststörungen und depressiven Störungen wird mit dem \textit{Diagnostik-System für psychische Störungen im Kindes- und Jugendalter nach ICD-10 und DSM-IV} geprüft \cite<DISYPS-KJ>{Doepfner:2003}. Die Eltern werden mit Hilfe des Fremdbeurteilungsbogens für Angststörungen (FBB-ANG) und für depressive Störungen (FBB-DES) den Schweregrad der Störungen und den Grad der psychosozialen Beeinträchtigung einschätzen (Problemstärke). Um das soziale ängstliche Verhalten aus Sicht des Kindes zu erfassen, wird die \textit{Social Anxiety Scale for Children - Revised - Deutsch} \cite<SASC-R-D>{Melfsen:1999}. Die SASC-R-D ist ein Verfahren zur Erfassung sozialer Angst bei Kindern und Jugendlichen im Alter von acht bis 16 Jahren, das sich aus den Skalen \textit{Furcht vor negativer Bewertung} und \textit{Vermeidung von und Belastung durch soziale Situationen} zusammensetzt.

\textbf{Operationalisierung.} TBD: 

\textbf{Intervention.} TBD: (->es werden noch Daten vom KJPD erwartet) Das Training ist vor allem für die Behandlung von Kindern entwickelt worden, die unter einer „emotionalen Störung mit Trennungsangst“, einer „Störung mit sozialer ängstlichkeit“, einer „Sozialen Phobie“ oder einer „generalisierten Angststörung“ leiden. Es richtet sich an Mädchen und Jungen im Alter von sieben bis zwölf Jahren sowie deren Eltern.

\textbf{Auswertungsmthoden.} TBD: Einfaktorielle Varianzanalyse mit Messwiederholung -> eine Gruppe, mind. 28 Teilnehmer, mittleren Effekt von f = 0.25, Signifikanzniveau von 5\% und einer Wahrscheinlichkeit von 80\% ausgegangen, bei vier Messzeitpunkten. Die Korrelation wird auf 0.4 gelegt, da davon ausgegangen wird, dass sich die Messungen zwischen den Versuchsteilnehmer unterscheidet, jedoch innerhalb der Versuchsperson einen hohe Korrelation besteht.

\textbf{Diskussion.} TBD: In zukünftigen Studien könnte die Zuverlässigkeit der Aussagen über das Verhalten der Kinder erhöht werden, in- dem zusätzlich Einschätzungen von anderen Bezugspersonen, wie beispielsweise Klassen- oder Fachlehrern, erhoben werden. In der vorliegenden Studie wurden für die Überprüfung der Wirksamkeit des Trainings nur Fragebögen eingesetzt. Um therapeutische Veränderungen noch zuverlässiger abbilden zu können, sollte in weiterführenden Studien eine multimethodale Datenerhebung, beispielsweise ergänzt um Interviewverfahren und Verhaltensbeobachtungen, gewählt werden. Darüber hinaus wird empfohlen, Katamnesestudien durchzuführen, um Aussagen über die Stabilität der erzielten Effekte treffen zu können (Ortbandt \& Petermann, 2009)