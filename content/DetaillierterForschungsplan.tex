%%%%%%%%%%%%%%%%%%%%%%%%%%%%%%%%%%%%%%%%%%%%%%%%%%%%%%%%%%%%%%%%%
%_____________ ___    _____  __      __ 
%\____    /   |   \  /  _  \/  \    /  \  Institute of Applied
%  /     /    ~    \/  /_\  \   \/\/   /  Psychology
% /     /\    Y    /    |    \        /   Zürcher Hochschule 
%/_______ \___|_  /\____|__  /\__/\  /    fuer Angewandte Wissen.
%        \/     \/         \/      \/                           
%%%%%%%%%%%%%%%%%%%%%%%%%%%%%%%%%%%%%%%%%%%%%%%%%%%%%%%%%%%%%%%%%
%
% Project     : Latex Vorlage SemArbeit
% Title       : 
% File        : einleitung.tex Rev. 00
% Date        : 24.10.2012
% Author      : Till J. Ernst
%
%%%%%%%%%%%%%%%%%%%%%%%%%%%%%%%%%%%%%%%%%%%%%%%%%%%%%%%%%%%%%%%%%
\chapter{Detaillierter Forschungsplan}\label{chap.forschungsplan}
\glsresetall
\textbf{Design.} Für die Studie wird ein experimentelles Design mit vier Vergleichsgruppen (Double-Pretest-Follow-up-Design) über den Zeitraum von zwei Jahren gewählt. Im KJPD werden zwei Gruppentherapien pro Jahr angeboten. Ziel ist es die Wirksamkeit der Gruppentherapie (Intervention) und deren Langzeitfolge zu untersuchen. Die Intervention besteht aus einem am KJPD Schaffhausen angebotenen Gruppentherapi-Angebot, das über den Zeitraum von vier Monaten an 12 Sitzungen stattfindet. Dabei findet für jede Gruppe vor und nach der Intervention (X) zwei Erhebungszeitpunkte (O\textsubscript{1-4}) statt. Die Messzeitpunkte (O\textsubscript{1 und 2}) vor der Intervention dienen der Absicherung für die Wirksamkeit bei den Wartegruppen. Der Messzeitpunkt (0\textsubscript{3}) unmittelbar nach der Intervention dient zur Wirksamkeitsüberprüfung der Gruppentherapie. Der vierte und letzte Messzeitpunkt(0\textsuperscript{3}) erfolgt ein halbes Jahr nach der Intervention und dient der Überprüfung, ob die gefundenen Effekte nachhaltig und sind.

\begin{center}
    \framebox{
        N\textsubscript{1-4} \hspace{3mm} O\textsubscript{1}    \hspace{3mm} O\textsubscript{2} \hspace{3mm} X \hspace{3mm} O\textsubscript{3} \hspace{3mm} O\textsubscript{4}
    }
\end{center}
N = nicht äquivalente Gruppen; O = Messung (=Observation); X Intervention

Für die Diagnose wird ein zweistufiges Vorgehen gewählt. Als erstes sollen die Eltern mit Hilfe der Diagnosecheckliste aus dem Training mit sozial unsicheren Kindern \cite{Petermann:2015} das Verhalten ihres Kindes gemäss den ICD-10 Kriterien beurteilen (Trennungsangst, soziale Ängstlichkeit, generalisierte Angststörung und depressive Stimmung). Anschliessend werden diejenigen Kinder, die den Diagnosekriterien einer Angststörung entsprechen, von den behandelnden Psychologinnen und Psychologen am KJPD exploriert. Wird die Diagnose bestätig, werden die Kinder in die Gruppe aufgenommen.




\textbf{Operationalisierung (8-tung: erst im Entwurfsstadium).} Für die Überprüfung der Wirksamkeit des Trainings wurden neben Angaben der Eltern (Fremdeinschätzung) auch Auskünfte der Kinder (Selbsteinschätzung) herangezogen. Die Datenerhebungen erfolgten unmittelbar vor und nach dem Training. 
Siehe \citeA{Ortbandt:2009}.


In zukünftigen Studien könnte die Zuverlässigkeit der Aussagen über das Verhalten der Kinder erhöht werden, in- dem zusätzlich Einschätzungen von anderen Bezugspersonen, wie beispielsweise Klassen- oder Fachlehrern, erhoben werden. In der vorliegenden Studie wurden für die Überprüfung der Wirksamkeit des Trainings nur Fragebögen eingesetzt. Um therapeutische Veränderungen noch zuverlässiger abbilden zu können, sollte in weiterführenden Studien eine multimethodale Datenerhebung, beispielsweise ergänzt um Interviewverfahren und Verhaltensbeobachtungen, gewählt werden. Darüber hinaus wird empfohlen, Katamnesestudien durchzuführen, um Aussagen über die Stabilität der erzielten Effekte treffen zu können (Ortbandt \& Petermann, 2009)