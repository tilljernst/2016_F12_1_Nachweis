%%%%%%%%%%%%%%%%%%%%%%%%%%%%%%%%%%%%%%%%%%%%%%%%%%%%%%%%%%%%%%%%%
%_____________ ___    _____  __      __ 
%\____    /   |   \  /  _  \/  \    /  \  Institute of Applied
%  /     /    ~    \/  /_\  \   \/\/   /  Psychology
% /     /\    Y    /    |    \        /   Zürcher Hochschule 
%/_______ \___|_  /\____|__  /\__/\  /    fuer Angewandte Wissen.
%        \/     \/         \/      \/                           
%%%%%%%%%%%%%%%%%%%%%%%%%%%%%%%%%%%%%%%%%%%%%%%%%%%%%%%%%%%%%%%%%
%
% Project     : Latex Vorlage SemArbeit
% Title       : 
% File        : einleitung.tex Rev. 00
% Date        : 24.10.2012
% Author      : Till J. Ernst
%
%%%%%%%%%%%%%%%%%%%%%%%%%%%%%%%%%%%%%%%%%%%%%%%%%%%%%%%%%%%%%%%%%
\chapter{Methode}\label{chap.methode}
\glsresetall
Erster Vorschlag für die Methode für unseren Nachweis:\newline
Prä-Post-Kontrollgruppendesign. 
Für die Überprüfung der Wirksamkeit des Trainings wurden neben Angaben der Eltern (Fremdeinschätzung) auch Auskünfte der Kinder (Selbsteinschätzung) herangezogen. Die Datenerhebungen erfolgten unmittelbar vor und nach dem Training. 
Siehe \citeA{Ortbandt:2009}.

Prä-Post-Follow-up-(Warte-)Kontrollgruppendesign. Unser Design könnte wie auf der von mir versendeter Grafik erweitert werden. Somit würden wir uns noch etwas von der Studie um \citeA{Ortbandt:2009} abgrenzen. Können wir aber gerne am kommenden Dienstag besprechen.

In zukünftigen Studien könnte die Zuverlässigkeit der Aussagen über das Verhalten der Kinder erhöht werden, in- dem zusätzlich Einschätzungen von anderen Bezugspersonen, wie beispielsweise Klassen- oder Fachlehrern, erhoben werden. In der vorliegenden Studie wurden für die Überprüfung der Wirksamkeit des Trainings nur Fragebögen eingesetzt. Um therapeutische Veränderungen noch zuverlässiger abbilden zu können, sollte in weiterführenden Studien eine multimethodale Datenerhebung, beispielsweise ergänzt um Interviewverfahren und Verhaltensbeobachtungen, gewählt werden. Darüber hinaus wird empfohlen, Katamnesestudien durchzuführen, um Aussagen über die Stabilität der erzielten Effekte treffen zu können (Ortbandt \& Petermann, 2009)
