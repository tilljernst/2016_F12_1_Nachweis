%%%%%%%%%%%%%%%%%%%%%%%%%%%%%%%%%%%%%%%%%%%%%%%%%%%%%%%%%%%%%%%%%
%  _____   ____  _____                                          %
% |_   _| /  __||  __ \    Institute of Computitional Physics   %
%   | |  |  /   | |__) |   Zuercher Hochschule Winterthur       %
%   | |  | (    |  ___/    (University of Applied Sciences)     %
%  _| |_ |  \__ | |        8401 Winterthur, Switzerland         %
% |_____| \____||_|                                             %
%%%%%%%%%%%%%%%%%%%%%%%%%%%%%%%%%%%%%%%%%%%%%%%%%%%%%%%%%%%%%%%%%
%
% Project     : LaTeX doc Vorlage für Windows ProTeXt mit TexMakerX
% Title       : 
% File        : diskussion.tex Rev. 00
% Date        : 7.5.12
% Author      : Remo Ritzmann
% Feedback bitte an Email: remo.ritzmann@pfunzle.ch
%
%%%%%%%%%%%%%%%%%%%%%%%%%%%%%%%%%%%%%%%%%%%%%%%%%%%%%%%%%%%%%%%%%

\chapter{Diskussion und Ausblick}\label{chap.diskussion}

\begin{itemize}
\item Eingehen auf die Gefahren von SM (auch wenn Umfang nicht reichte, dies in der Arbeit einfliessen zu lassen), ev Buch von Digitaler Demenz erwähnen
\item Weiterführende Arbeit spezifische Auseinandersetzung mit Narzissmus und SM
\item Weiterführende Studien zu Social-Capital, Selfpresentation, Self-Disclosure und Satisfaction im Allgemeinen zu SWB. Nur oberflächlichen Zugang
\item Weitere Forschung im Bereich von Virtual Role Playing Games
\end{itemize}
