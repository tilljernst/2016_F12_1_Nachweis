%%%%%%%%%%%%%%%%%%%%%%%%%%%%%%%%%%%%%%%%%%%%%%%%%%%%%%%%%%%%%%%%%
%  _____   ____  _____                                          %
% |_   _| /  __||  __ \    Institute of Computitional Physics   %
%   | |  |  /   | |__) |   Zuercher Hochschule Winterthur       %
%   | |  | (    |  ___/    (University of Applied Sciences)     %
%  _| |_ |  \__ | |        8401 Winterthur, Switzerland         %
% |_____| \____||_|                                             %
%%%%%%%%%%%%%%%%%%%%%%%%%%%%%%%%%%%%%%%%%%%%%%%%%%%%%%%%%%%%%%%%%
%
% Project     : LaTeX doc Vorlage für Windows ProTeXt mit TexMakerX
% Title       : 
% File        : header.tex Rev. 00
% Date        : 23.4.12
% Author      : Remo Ritzmann
% Feedback bitte an Email: remo.ritzmann@pfunzle.ch
%
%%%%%%%%%%%%%%%%%%%%%%%%%%%%%%%%%%%%%%%%%%%%%%%%%%%%%%%%%%%%%%%%%

\documentclass[
	oneside,
	12pt,
	parskip=half,
	ngerman,
	%bibtotocnumbered,
	%bibliography=totocnumbered,
	%listof=totoc, %Abbildungs- und Tabbellenverzeichnis ins Inhaltsverzeichnis aufnehmen
	version=first
	%liststotoc
]{scrreprt}
%article scrartcl
%report scrreprt
%book scrbook
%letter scrlttr2

%***********************************************************************
% include some libs
%***********************************************************************	
\usepackage[utf8]{inputenc}
\usepackage{listings}
\usepackage{color}
\usepackage{fancyhdr}
\usepackage{rotating}
\usepackage{titlesec}
\usepackage{mathptmx} 
\usepackage[scaled=.90]{helvet}
\usepackage{courier}
\usepackage{setspace} %Zeilenabstand
\onehalfspacing % 1,5 Zeilenabstand
%\renewcommand*\familydefault{\sfdefault} %% Only if the base font of the document is to be sans serif
\usepackage[T1]{fontenc}
\usepackage{german}
%\usepackage[ngerman]{babel}
%\usepackage[babel]{csquotes}
\usepackage{textcomp}
\usepackage[squaren]{SIunits}
\usepackage{graphicx}
\usepackage{url}
\usepackage{geometry}
\usepackage[absolute]{textpos}
\usepackage{makeidx}
\usepackage{colortbl}
\usepackage{pdflscape}
\usepackage{pdfpages}
\usepackage{tabularx}
\usepackage{lmodern}
\usepackage{longtable}
\usepackage{multirow}
\usepackage{array}
\usepackage{float}
\usepackage{scrhack}
\usepackage{wallpaper}
\usepackage{titleref}

% Bereitstellung Hyperlinkfunktionen (PDF) (muss als letztes Paket geladen werden)
\usepackage[
	colorlinks=true,
	breaklinks=true,
	linkcolor=black,
	citecolor=black,
	urlcolor=black,
	anchorcolor=black,
	pdfpagelabels,
	pdftitle={Latex Vorlage SemArbeit},
	pdfsubject={Facebook und der Glücksfaktor},
	pdfkeywords={KEYWORDS},
	pdfauthor={Till J. Ernst (ernsttil)}
]{hyperref}

\usepackage{apacite}

% biblatex-apa
%\usepackage[style=apa,
%			backend=biber,
%			babel=hyphen,
%			natbib=true
%			style=authoryear, 
%    		maxcitenames=2, 
%    		sorting=nyt,
%    		backref=true
%]{biblatex}
%\DeclareLanguageMapping{ngerman}{{ngerman-apa}} 
%\addbibresource{Literatur} %for biblatex


% Bereitstellung Glossar
%\usepackage[toc]{glossaries} %fügt glossarie zum Inhaltsvz hinzu
\usepackage{glossaries}
%\usepackage[acronym]{glossaries}
\makeglossaries





%***********************************************************************
% various styles
%***********************************************************************	

%create index
\makeindex

%define pagestyle
\pagestyle{fancy}

%use sans-serif font 
%\renewcommand{\familydefault}{\sfdefault}

%define page margin
\geometry{a4paper, top=25mm, left=25mm, right=25mm, bottom=25mm,headsep=10mm,footskip=10mm}

%textpos parameter
\setlength{\TPHorizModule}{30mm}
\setlength{\TPVertModule}{\TPHorizModule}
\textblockorigin{10mm}{10mm} % start everything near the top-left corner
%The \setlength command is used to set the value of a length command, len-cmd, which is specified as the first argument.
\setlength{\headheight}{28pt}
\setlength{\parindent}{0pt} % Bei Absatz neuer Einzug

%horizontal lines for titlepage 
\newcommand{\HRule}{\rule{\linewidth}{0.5mm}}

%reference to source items inlc source number
\newcommand{\srcref}[1]{\nameref{src:#1} \cite{#1}}

% Umdefinieren des Layouts (otional)
% - - - - - - - - - -
\fancyhf{} %alle Kopf- und Fußzeilenfelder bereinigen
%\fancyhead[OR,EL]{\rightmark} %die Section-Name
\fancyhead[L]{\leftmark} % Chapter-Name
%\fancyhead[OL,ER]{\leftmark} % Chapter-Name -> nur nötig bei 2 Seiten Layout
\renewcommand{\headrulewidth}{0.3pt}
\fancyfoot[C]{\textbf \Large \thepage} 
\renewcommand{\footrulewidth}{0pt}
\renewcommand{\arraystretch}{1.5} % für tabellen

%\fancyhead[LO,RE]{} %clear headings for contents 
%\fancyhead[RO,LE]{\nouppercase{\rightmark}} %right odd pages and left even pages
%\fancyhead[LO,RE]{\MakeUppercase{\leftmark}} %left odd pages and right even pages
%\fancyfoot[LE,RO]{\thepage} %page numbering
%\fancyfoot[C]{} %clear centered page numbering 

%define some colors
\definecolor{gray}{rgb}{0.95,0.95,0.95}
\definecolor{darkgray}{rgb}{0.4,0.4,0.4}
%listing colors
\definecolor{lgray}{RGB}{250,250,250}
\definecolor{lgreen}{RGB}{63,127,95}
\definecolor{lred}{RGB}{127,0,85}
\definecolor{lblue}{RGB}{42,0,255}

%***********************************************************************
% listing
%***********************************************************************

\lstset{		
		basicstyle=\small\ttfamily,
		frame=single,
		numbers=left,	
		numberstyle=\tiny,
		%firstnumber=auto,
		numberblanklines=true,
		captionpos=b,
		extendedchars=true,
		float=ht,
		showtabs=false,
		tabsize=2,
		showspaces=false,
		showstringspaces=false,
		breaklines=true,
		%prebreak=\Righttorque,
		backgroundcolor=\color{lgray},
		keywordstyle=\color{lred}\bfseries, 
		commentstyle=\color{lgreen}\ttfamily,
%		morekeywords={printstr, printhexln},
		stringstyle=\color{lblue},
		xleftmargin=0.5cm,
		xrightmargin=0.5cm
}

%\lstloadlanguages{C++}

%\lstdefinelanguage{xc}{
%     keywords={printstr, printhexln, attributes, class, classend, do, empty, endif, endwhile, fail, function, functionend, if, implements, in, inherit, inout, not, of, operations, out, return, set, then, types, while, use},
%     keywordstyle=\color{lred}\bfseries,
%     ndkeywords={},
%     ndkeywordstyle=\color{yellow}\bfseries,
%     identifierstyle=\color{black},
%     sensitive=false,
%     comment=[l]{//},
%     commentstyle=\color{lgreen}\ttfamily,
%     string=[l]{"},
%     stringstyle=\color{lblue}\ttfamily
%  }
